\part{Introduction}
\label{part:part01}

\begin{partintro}
\lettrine{T}{his} is the first part of the book.

Provide an overview of what this part covers and why it matters. Set the context for the chapters that follow.

\vspace{1em}

\textbf{In this part:}
\begin{itemize}
\item Chapter 1: Opening concepts
\item Chapter 2: Core material
\item Chapter 3: Advanced topics
\end{itemize}
\end{partintro}

\chapter{First Chapter}
\label{ch:chapter01}

\begin{chapterintro}
Introduction to the first chapter. Explain what the reader will learn and why it's important.
\end{chapterintro}

\section{Introduction}

Start with the basics. Provide context and motivation.

\section{Main Content}

Core material goes here. Break it down into digestible sections.

\subsection{Important Concepts}

Explain key concepts clearly.

\begin{keyidea}
Highlight the most important idea that readers should remember from this section.
\end{keyidea}

\subsection{Details}

Provide detailed explanations with examples.

\begin{example}
Concrete example that illustrates the concept.
\end{example}

\section{Summary}

Summarize what was covered and prepare for the next chapter.

\begin{note}
Additional notes or considerations that don't fit in the main text.
\end{note}